% Metódy inžinierskej práce

\documentclass[10pt,twoside,slovak,a4paper]{article}

\usepackage[slovak]{babel}
%\usepackage[T1]{fontenc}
\usepackage[IL2]{fontenc} % lepšia sadzba písmena Ľ než v T1
\usepackage[utf8]{inputenc}
\usepackage{graphicx}
\usepackage{url} % príkaz \url na formátovanie URL
\usepackage{hyperref} % odkazy v texte budú aktívne (pri niektorých triedach dokumentov spôsobuje posun textu)

\usepackage{cite}
%\usepackage{times}

\pagestyle{headings}

\title{Návrh zamerania projektu
\thanks{Semestrálny projekt v predmete Metódy inžinierskej práce, ak. rok 2025/26, vedenie: PaedDr. Pavol Baťalík}}

\author{Denis Gaál, Kira Galat, Marek Gruľa, Miroslav Hadbavný\\[2pt]
	{\small INTUTOR}\\
	{\small Slovenská technická univerzita v Bratislave}\\
	{\small Fakulta informatiky a informačných technológií}\\
	{\small \texttt{...@stuba.sk}}
	}

\date{\small 6. október 2025} % upravte



\begin{document}

\maketitle

\section{Názov projektu}
Inteligentný tutor pre stredoškolákov

\section{Akronym projektu}
INTUTOR

\section{Anotácia}
Projekt INTUTOR sa zameriava na vývoj inovatívnej webovej aplikácie založenej na umelej inteligencii (AI), ktorá slúži ako interaktívny tutor pre stredoškolských žiakov v predmete informatika. Cieľom je zlepšiť učenie programovania a základných konceptov informatiky prostredníctvom hravej a personalizovanej formy. Aplikácia bude využívať pokročilé modely AI (napr. veľké jazykové modely ako GPT alebo podobné) na generovanie vysvetlení, interaktívnych kvízov, programovacích úloh a personalizovaných odporúčaní prispôsobených úrovni znalostí žiaka. Zameranie zahŕňa: 1) Analýzu úrovne žiaka na základe interakcií a testov; 2) Generovanie dynamického obsahu pre témy ako algoritmy, dátové štruktúry, objektovo-orientované programovanie a základné webové technológie; 3) Integráciu gamifikácie (body, úrovne, odmeny) na zvýšenie motivácie; 4) Testovanie v reálnom školskom prostredí na vybraných stredných školách v SR. Výstupom bude funkčný prototyp aplikácie s otvoreným kódom, dostupný pre školy, s potenciálom rozšírenia na iné predmety. Projekt prispeje k digitalizácii vzdelávania, zlepšeniu digitálnych zručností mládeže a riešeniu nedostatku kvalifikovaných IT špecialistov v SR. Očakávaná TRL úroveň: 6-7 (prototyp testovaný v operačnom prostredí).

\end{document}
